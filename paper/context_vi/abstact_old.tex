Thị trường chứng khoán là nơi phản ánh rõ các hoạt động kinh tế của một quốc gia và các mối quan hệ quốc tế. Đã có rất nhiều
lực nhằm tìm ra quy luật của thị trường chứng khoán nhằm đưa ra các công thức tổng quát cho mỗi mã cổ phiếu tương ứng.
Nhưng hộp đen chứng khoán có rất nhiều yếu tố ảnh hưởng không chỉ đơn giản dựa vào lịch sử giá. Nhưng dựa trên lịch sử gía cổ phiế
để đưa ra các phân tích kỹ thuật có thể giúp nhà đầu tư dự đoán được phần nào.
Có nhiều phương pháp phân tích kỹ thuât cổ phiếu đã được ra đời từ những năm 1960. Các phương pháp từ thời đó phần lớn dựa trên các công thức toán học tường minh.
Các phương pháp cổ không có nhiều điều bí ẩn nhưng kết quả đat được chưa như mong đợi.
Kể từ năm 2010, máy học đã đạt đươc những đôt phá nhất định. Ý tưởng chinh phục cổ phiếu đã quay trở lại và mãnh liệt hơn khi học sâu đã vượt qua hàng loạt các thuật toán khác chứng tỏ sự yêu việc của mình. Đặc biệt khi sử dụng các mạng thần kinh tuần tự cho các vấn đề chuỗi thời gian cho kết quả chính xác ngoài mong đợi
Nhưng việc chọn mô hình và đưa ra các siêu tham số là một công viêc cần có kinh nghiệm chuyên sâu về cả cổ phiếu lẫn máy học. 
Không phải ai cũng có khả năng đó. Vì vậy chúng tối đễ xuất tối sử dụng tối ưu hoá bayesian và tìm cách xây dựng mô hình sử dụng LSTM dễ tạo các mô hình có nhiều siêu tham số để tối ưu để mô hình đưa ra